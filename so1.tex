\documentclass[a4j,11pt]{jsarticle}
\usepackage{semi}

\makeatletter % プリアンブルで定義開始

% 図番号を"<章節などの番号番号> - <図番号>" へ
\renewcommand{\thefigure}{\thesubsection-\arabic{figure}}
\renewcommand{\thetable}{\thesubsection-\arabic{table}}
% 章が進むごとに図番号をリセットする
\@addtoreset{figure}{subsection}
\@addtoreset{table}{section}

\makeatother % プリアンブルで定義終了

\renewcommand{\headfont}{\bfseries} %章タイトルなどを明朝体にする

\makeindex
\begin{document}
\definecolor{cellcolor}{rgb}{ 1, .90, .90}
\definecolor{rowcolor}{rgb}{.85, .85, 1}
\setcounter{tocdepth}{3}
\thispagestyle{empty}
\begin{center}

\huge
2020年度 卒業論文\\[60pt]
\HUGE
山田太郎物語とその続き\\
主演:なすび\\[65pt]
\huge
指導教員 須田 宇宙 准教授\\[40pt]
千葉工業大学 情報ネットワーク学科\\[10pt]
須田研究室\\[40pt]
1632888 \hspace{70pt} 田中 中田\\[110pt]
\end{center}
\begin{flushright} 
\huge

\textcolor{white}{文字}

\textcolor{white}{文字}

提出日 2020年1月25日
\end{flushright}
\newpage
\thispagestyle{empty}
\large
% 目次
\tableofcontents

\newpage
%表目次
\listoftables
%図目次
\listoffigures

\begin{comment}

\end{comment}
\newpage


\newpage

\subsection{提案手法}
\subsection{ケプストラム法}
\subsection{逆畳み込み}

\newpage

\section{実験と検証}
\subsection{調査目的と調査方法}
\subsection{回答の集計と効果の検証}
\subsection{考察}

\newpage
\section{結言}

\newpage
\section*{謝辞}
\addcontentsline{toc}{section}{謝辞}
ここに研究の謝辞.主にご協力いただいた方など.

\newpage
\bibliographystyle{jplain}
\addcontentsline{toc}{section}{参考文献}
\begin{thebibliography}{99}

\bibitem{oka1}国土技術センター,"自然災害の多い国",http://www.jice.or.jp/knowledge/japan/commentary09
\bibitem{oka2}佐藤 由希子,"インパルス応答を用いた反射音除去のための一提案", 千葉工業大学卒業論文, 2014
\bibitem{oka3}小泉 宣夫,"基礎音響・オーディオ学",コロナ社 2005 36-38頁

\end{thebibliography}

\newpage
\section*{付録}
\addcontentsline{toc}{section}{付録}

\end{document}
